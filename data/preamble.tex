%Defines Unicode characters according to UTF8 standard (basically allows you to writ any non English ASCII character)
\usepackage[utf8]{inputenc}

\usepackage{longtable}

\usepackage{tabularx}

%Adds 8-bit font encoding to the document and adds the charter font to be used throughout 
\usepackage[T1]{fontenc}
\usepackage{charter}

%Bibliography handling (Read this if you need help: https://www.overleaf.com/learn/latex/Articles/Getting_started_with_BibLaTeX)
\usepackage[english]{babel}
\usepackage[backend = biber, style = authoryear, sorting = nty, citestyle = authoryear]{biblatex} %Change style here. Tho i prefer this method of citation ;)


%Allows for determining margins of the document
\usepackage{geometry}

%Allows easier handling of author and affiliations
\usepackage{authblk}

%Packages for assisting in displaying mathematics
\usepackage{amsmath,amsthm,amssymb}

%Package for creating "fancy" headers and footers
\usepackage{fancyhdr}

%Allows for the document to import images 
\usepackage{graphicx}

%Package for forcing placement of Objects inside the current section 
\usepackage{placeins}

%Allows for wrapping text around figures
\usepackage{wrapfig}

%Package for options on captions
\usepackage{caption}
\usepackage[labelfont={it, small},textfont={it, small}]{caption}

\usepackage{subcaption}

%Package for displaying and highlighting code
\usepackage{minted}

%Allows for changing certain attributes related to color
\usepackage{xcolor}

%Allows you to reference the final page
\usepackage{lastpage}

% Add bibliography and index to the table of contents. The [nottoc] skips the table of contents so it doesn't refer to itself
\usepackage[nottoc]{tocbibind}

%Quote handling 
\usepackage{csquotes}

%allows us to import PDF files
\usepackage{pdfpages}

%Google sheet
\usepackage{booktabs, multirow} % for borders and merged ranges
\usepackage{soul}% for underlines
\usepackage{changepage,threeparttable} % for wide tables

\setlength{\parindent}{0pt} %no indents

%allows for writing c# code in overleaf
\usepackage{listings}
\usepackage{xcolor}
\lstdefinestyle{sharpc}{language=[Sharp]C, frame=lr, rulecolor=\color{blue!80!black},keywordstyle=\color{magenta},numbers=left}



%Allows for making hyperlinks in the text !(HAS TO BE LAST IMPORT)!
\usepackage{hyperref}

\usepackage{titlepic}

\usepackage{graphicx}


